\documentclass[12pt,a4paper]{article}
\usepackage{amsmath,graphicx}
\setlength{\oddsidemargin}{0in}
\setlength{\textwidth}{6in}
\setlength{\textheight}{9in}
\begin{document}

%%%%
%%%% You can either make a titlepage, like this:
%%%%
%\begin{titlepage}
%\begin{center}
%{\Huge The Traveling Salesman Problem}\\[.5cm]
%{\Large S. Bose and A. Einstein}\\[.5cm]
%{\it Department of Mathematical Physics,\\ 
%National University of Ireland, Maynooth, Ireland}\\[2cm]
%\end{center}
%\begin{abstract}
%
%Here we write out the abstract for the project. 
% 
%\end{abstract}
%
%\end{titlepage}

%%%%
%%%% or you can use \maketitle, like this:
%%%%
\title{ Travelling Salesman}
\author { S. Bose and A. Einstein}
%\address{Department of Mathematical Physics, National University of
%  Ireland Maynooth, Maynooth, Co Kildare, Ireland}
%%%% (annoyingly, there is no \address command in the standard
%%%% article class!)
\date{\today}
\maketitle

\begin{abstract}

Here we write out the abstract for the project. 
 
\end{abstract}
%\maketitle                 
%%%
%%%Introduction
%%%
\section{Introduction}

Here you can pop in sections which are numbered. You can also make
text in {\em italics}, and in {\bf bold face}. There are two main
types of text environments:
\begin{enumerate}
\item ordinary text and
\item math text.
\end{enumerate}
In
ordinary text one can have a bulleted list by starting an itemize
environment, or a numbered list by starting an enumerate environment
\begin{itemize}
\item This is bullet item 1
\item This is bullet item 2
\end{itemize}
For the math text, there are two types, inline math which is done like
$\psi=\sqrt{\alpha+\beta}$, and standalone equations which are done
like
\begin{equation}
\psi=\sqrt{\alpha+\beta}
\end{equation}
There are lots of special math symbols. I have copied a small chart
which summarises the various math symbols. Another handy equation
environment is the align environment. This is useful if you
have a number of equations that you want to write down beneath each
other:
\begin{align}
\psi_1&=e^{-x^2/2}\\
\psi_2&=e^{-x^2/4}
\end{align}
here the ampersands make sure the equal signs in the equations all
line up vertically.

One can also have unnumbered sections:
\section*{Unnumbered Section}

and also have numbered and unnumbered subsections
\subsection{A little subsection}

All standalone equations are automatically numbered. You can introduce
a name for any standalone using the label command:
\begin{equation}
\psi_{i,j}^{n+1}=\frac{1}{4}\left( \psi_{i+1,j}+\psi_{i-1,j}+\psi_{i,j-1}+\psi_{i,j+1}\right)\;\;,
\label{jacobi}
\end{equation}
%
This equation is now labelled by the symbol {\bf jacobi} and one can
refer to the automatically generated equation number for this equation
by (\ref{jacobi}). If you stick in equations before the jacobi
equation, the equation numbers will alter but the symbol
\eqref{jacobi} will always point to the jacobi equation. One can also
introduce citation references. Again these are given internal symbols
which, when you latex the document, are substitutted for numbers. For
instance, to reference work \cite{Einstein}, we will have in the
bibliography at the end of this document a bibliography item which
will be denoted by the symbol Einstein. When you are finished and
latex the document, \LaTeX will work out the appropriate number to
insert into the in-text citation eg. \cite{Einstein} or \cite{Bose}.

\section{Figures and tables}
\label{figsandtabs}

Figures and tables are {\em floating bodies}, which means that \LaTeX
will put them where there is room for them, rather than exactly where
you put them in the text.

\subsection{Figures}
\label{figs}

To insert pictures you must produce encapsulated postscript files of
the relevant matlab figure. Lets say we have in our directory a saved
figure called wavefunction.eps. Then the following pice of \LaTeX will
insert the figure into our document:

\begin{figure}[htp]
\begin{center}
\includegraphics[width=8cm]{wavefunction.eps}
\end{center}
\caption{This is the caption for the figure}
\label{fig1}
\end{figure}

You also notice that we have labelled the figure so that we can refer
to it in the text as Figure \ref{fig1}.  The {\tt [htp]} inside the
square brackets tells \LaTeX to try to put the figure here[h] if
possible, or at the top[t] of the page, or otherwise on a separate
page[p]. 

\subsection{Tables}
\label{tabs}

You can insert tables in a similar way to how figures were inserted in
section \ref{figs}:

\begin{table}[hbt]
\begin{center}
\begin{tabular}{|c|llc|}
No of towns & $l_{\min}$ & attempts & time\\ \hline
20 & 3.8752 & 10 & 20min \\
50 & 9.3821 & 40 & 3hrs \\
100 & 24.623 & 100 & 9d \\
200 & 67.342 & 500 & 7yr
\end{tabular}
\end{center}
\caption{Results for the travelling salesman with a random
  configuration of towns. $l_{\min}$ is the shortest
roundtrip found in the given number of attempts.}
\label{tab:results}
\end{table}

\begin{thebibliography}{99}
\bibitem{Bose} S. Bose, Phys. Rev. A {\bf 91}, p. 1121 (1928).
\bibitem{Einstein} A. Einstein, Phys. Rev. {\bf 1}, p. 1 (1928).
\end{thebibliography}


\end{document}
